\section{Conférence 8 : VRAC Alexandre, Chef de projet informatique Temps Réels chez Safran Engineering Services}
	
 	\subsection{Présentation des conférencier}
		Le conférencier est  nommé VRAC Alexandre. Chef de projet informatique Temps Réels chez Safran Engineering Services. Il a un parcours académique particulier, il commence par un DEUG MIAS dans l'université paris 8, puis un DUT en Génie Mathématique et Informatique à Paris 5, puis un IUP2 MIME à paris 8 et enfin en 2007 un Master IMA encore à paris 8.

	\subsection{Présentation des projets présentés}
		$\rightarrow$ Siemens (2007/2008):
		\begin{itemize}
			\item[-] ingénieur développeur logiciel 
			\item[-] logiciel dans le domaine médicale 
			\item[-] algorithme de migration de données
			\item[-] transcription de modules inter-langages
		\end{itemize}

		$\rightarrow$ Safran Sagem (2008/2012):
		\begin{itemize}
			\item[-] développement logiciel embarqué  
			\item[-] modulation de la munition d'armement 
			\item[-] développement d'un simulateur 
			\item[-] implémentation d'un nouveau système de guidage laser pour les munition.
		\end{itemize}

		$\rightarrow$ Thales System Aeroport (2012) 6 mois:
		\begin{itemize}
			\item[-] développement d'un logiciel embarqué 
			\item[-] évolution du logiciel de téléchargement des dates de l'équipement SPECTRA sur le RAFAL. 
			\item[-] modification du protocole de chargement des données 
		\end{itemize}

		$\rightarrow$  Zodiac Aerospace (2012):
		\begin{itemize}
			\item[-] implémentation d'un logiciel de vérification de la configuration du système de distribution électrique de l'airbus .
			\item[-] évolution du logiciel de téléchargement des dates de l'équipement SPECTRA sur le RAFAL. 
			\item[-] modification du protocole de chargement des données 
		\end{itemize}

		$\rightarrow$  Sagem group(2015):
		\begin{itemize}
			\item[-]Project Manager Officer 
			\item[-] Suivi et pilotage de l'avancement des sociétés du groupe Safran pour la substitution de substances interdites par une loi européenne(REACH) 
		\end{itemize}

		$\rightarrow$  Sagem Defense(actuellement):
		\begin{itemize}
			\item[-] validation de code du COPS du nouveau RAFAL.
		\end{itemize}



	\subsection{Technologie utilisée}
	Les technologies les plus utilisées par monsieur VRAC sont le langage C , le sched/SymuLink, et Java.
	\subsection{Avis personnel} 
	Une présentation très intéressantes car nous avions un ingénieur qui travaillait sur des armes de guerres. De plus, nous avons été introduit aux problèmes qu'amène l'informatique critique ainsi que le respect de normes .   