 \section{Conférence 4 : BELACEL Reda Ingénieur d'étude/développement et consultant fonctionnel technique MOE chez Sopra Stéria }
 
 	\subsection{Présentation du conférencier}
 Le conférencier est  monsieur BELACEL Reda. Il est à ce jour Ingénieur R\&D\footnote{Recherche et Développement} et consultant fonctionnel technique chez Sopra Steria\footnote{SSII international, née de la fusion de deux SSII Sopra et Steria, répartie sur 20 pays, comportant plus de 35000 collaborateurs}. Lors de cette présentation il nous explique qu'il en ce moment en mission chez ERDF\footnote{Électricité Réseau Distribution France} en tant que consultant fonctionnel technique .

 	\subsection{Présentation de l'entreprise}
 	La présentation à débuté par une présentation de l'entreprise dans laquelle monsieur BELACEL a été envoyé .Ainsi, nous sommes informés que ERDF est une société anonyme à conseil de surveillance et directoire, filiale à 100\% d'EDF chargée de la gestion de 95\% du réseau de distribution d'électricité en France.

 	\subsection{Présentation du projet courant}
 	La mission consiste en un projet de GMAO\footnote{Gestion de Maintenance assisté par ordinateur} pour les postes sources et un autre GMAO pour le réseaux électriques. Il nous explique s'il s'agit de l'adaptation d'une solution existantes (reprise d'un logiciel d'IBM le maximo 7.5.0 \& maximo 7.6.0) aux besoins de ERDF. Après nous avoir énoncé la problématique du projet, monsieur BELACEL nous énonce quelles ont été ces tâches lors de ce projet :
 	\begin{itemize}
 		\item génération de rapport avec BIRT\footnote{The Business Intelligence and Reporting Tool} 
 		\item Le développement d'une interface 
 		\item développement de scripts d'automatisation et de récupération de données en Python 
 		\item Administration de serveur Linux 
 		\item Rédaction des spécifications fonctionnelles ainsi que des PTI\footnote{Pratique des techniques Informatiques}    
 	\end{itemize} 	

 	\subsection{Technologie utilisée} % (fold)
 		Lors de cette conférence il nous a été présenté les différentes technologie utilisé par monsieur BELACEL. Certaines technologies était connue, tels les langages Java et Python,d'autres tel que le logiciel BIRT était nouvel et vraiment intéressantes . 
 	% subsection subsection_name (end)

 	\subsection{Avis personnel} 
 	Cette conférence est était très riche en nouveau terme, de plus il nous à été présenté une nouvelle technologie rendant cette présentation très intéressante . 