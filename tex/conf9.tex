\section{Conférence 9 : BEGREDJ Khaled, Chef de projet en établissement public}
 	\subsection{Présentation des conférencier}
 		Le conférencier est  nommé BEGREDJ Khaled. Il est à ce jour Chef de projet en établissement public . Il a eu son master II ISE à paris 8 de 2007 à 2008. Ensuite, Il devient ingénieur en développement chez XL Airways. 
 		Actuellement, il travaille au CSTB\footnote{Centre Scientifique et technique du Bâtiment}   
	\subsection{Présentation de l'entreprise}
	Le CSTB est un centre semi-privé qui recherche et expertise dans le bâtiment. Il rassemble , développe, partage avec les secteur de la construction .
	\subsection{Présentation du projet présenté}
	Le projet actuelle du conférencier est nommé RT-2012\footnote{RT: Régulation Thermique}, il s'agit d'un projet de d'évaluation et de calcul des besoins énergétiques . Sa tache est de diriger ce projet .
	\subsection{Technologies Présenté}
	Lors de cet conférence il nous a été présenté de nombreuses technologies. Les plus intéressantes : 
	\begin{itemize}
	 	\item les plates-formes collaborative tel que Alfresco
	 	\item Les bugs trackers tel Trac 
	 	\item les technologies d'intégration continue comme Jenkins
	 	\item les technologies d'inspection continue par exemple Sonar
	 	\item les outils de planification tel que Projector
	 \end{itemize} 
	\subsection{Avis personnel} 
	La présentation était très riche en nouvel technologie permettant la gestion d'un projet . Je me suis fait un plaisir de tout essayer .