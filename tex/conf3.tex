\section{Conférence 3 : PEYSSONNAUX Valérie Chef de projet chez ADP}
 	\subsection{Présentation du conférencier}
 La conférencière est nommée Valérie PEYSSONNAUX. Elle est à ce jour chef de projet chez ADP\footnote{Aéroport De Paris, entreprise française qui construit, aménage et exploite des plates-formes aéroportuaires, dont les deux principales, en France, sont celles des aéroports de Paris-Orly et de Paris-Charles-de-Gaulle.}. Cette conférence a pour but de nous sensibiliser au métier de la conduite de projet .
 	Dans un premier temps la conférencière nous présente son parcours sous forme d'un flèche chronologique. De manière résumé, elle commence par un BTS 1988, puis en 1992 elle commence on sa vie d'informaticien par de l'assistance technique par téléphone. Dans la même année, elle fait ses premiers projets de PAO\footnote{Publication Assisté par Ordinateur}, elle continue en faisant de la gestion de budget pour un projet, et ainsi finit dans le développement d'application . Lors années suivantes  les compétences de Madame PEYSSONNAUX se sont vue évoluer, elle pris un poste de d'analyste programmeur. Ainsi elle continuait à développer  tout en étant adjoint chef de projet, accumulant ainsi de l'expérience pour son poste présent de chef de projet .
  	\subsection{Présentation de l'entreprise}
  	ADP est une entreprise française qui construit, aménage et exploite des plates-formes aéroportuaires, dont les deux principales, en France, sont celles des aéroports de Paris-Orly et de Paris-Charles-de-Gaulle. Madame PEYSSONNAUX travail dans la branche DSI\footnote{Département des Systèmes d'information} et plus précisément dans la branche DSICW\footnote{Département des Systèmes d'information, Centre de Compétence , Web et mobilité} . 

  	\subsection{Présentation de l'entreprise} 
  	\label{sub:presentation_de_l_entreprise}
  	Aéroports de Paris (ADP) est une entreprise française qui construit, aménage et exploite des plates-formes aéroportuaires, dont les deux principales, en France, sont celles des aéroports de Paris-Orly et de Paris-Charles-de-Gaulle.
	Elle est le deuxième groupe aéroportuaire européen en termes de chiffre d'affaires aéroportuaire, après British airports authority (BAA), et le premier groupe européen pour le transport de fret et de courrier. Elle compte 189 compagnies aériennes clientes dont les acteurs majeurs du transport aérien. En 2011, son chiffre d'affaires s'est élevé à plus de 2,5 milliards d'euros et elle a accueilli sur ses plates-formes 88,1 millions de passagers.

 	\subsection{Présentation du métier de chef de projet } 
 	Cette présentation nous a exposés les principales problématique d'un chef de projet :
 		\begin{itemize}
 	  	\item Organiser et coordonnées l’équipe de projet .
 	  	\item Veiller au respect des délais, des coûts, et de la qualité .
 	  	\item Assurer l'interface avec la MOA\footnote{maîtrise d'ouvrage :  responsable de l’efficacité de l'organisation et des méthodes de travail autour des systèmes d'information} .
 	  	\item Contrôler les exigences clients .
 	  	\item Accompagner le projet .
 	  	\item Anticiper les évolutions des besoins du client .  
 	  	\end{itemize}  

 	\subsection{Question/Réponse}
 		\subsubsection{Avec toutes cet expériences pourquoi avoir fait un M2 à Paris 8?} 
 		\label{ssub:avec_toutes_cet_experiences_pourquoi_avoir_fait_un_m2}
 		Dans un premiers temps pour la reconnaissance qu'apporte un Master . Puis pour mon évolution professionnel, c'est à dire que sans un BAC+5 on a quasiment pas de possibilité de monté en grade dans une entreprise . Enfin un chef de projet doit savoir "faire-faire" tout en gardant le "savoir faire", de plus ce master m'as permis d'approfondir ma compréhension de la méthodologie d'un projet et donc me réactualiser .

 	\subsection{Avis personnel} 
 	Cette présentation avait pour but de nous faire comprendre les enjeux et les finalités du travail de chef de projet . Je pense que le but a été atteint . La présentation a été très complète et les termes techniques ont été définit . Le seul bémol serait que la présentation est peut-être trop riche .