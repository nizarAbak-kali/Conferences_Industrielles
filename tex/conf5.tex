\section{Conférence 5 : CAGNET Cyril, Responsable informatique des tris bagages chez ADP}
 	\subsection{Présentation du conférencier}
 		Le conférencier est  nommé CAGNET CYRIL. Il est à ce jour responsable informatique des tris bagages chez AdP\footnote{Aéroport de paris}. Son parcours est atypique car il est à quitté paris 8 après avoir obtenue sa licence pour partir travailler dans sa propre boite avec un autre ancien de paris 8. Suite à une mauvaise gestion ils auraient fait faillite. Et c'est ainsi en 2001 que le conférencier c'est retrouvé à donner des cours de .NET pendant 15 ans.Ensuite il est embauché chez AdP afin de gagner de l'expérience et de l'argent. Finalement, monsieur CAGNET et son partenaire ont repris l'entreprise spécialisé en C\# et .NET qui semble marché . 
 	\subsection{Présentation de l'entreprise}
 		cf. la conférence 3 pour la présentation de l'entreprise AdP. 
 	\subsection{Présentation du projet courant}
 		Le projet courant présenté consiste en la maintenance des  systèmes de tri bagages des trois terminaux de Charles-de-Gaulle. Dans un premier temps, le conférencier nous a présenté en détaille chaque terminal ainsi que la complexité de la tache qui est de faire fonctionner. \\
 		Présentation de CDG1:
 		\begin{itemize}
 			\item début de la réhabilitation du système de tri des bagages en 2006 . 
 			\item tout bagage est sécurisé et trié .
 			\item 15 000 bagages par jour .
 			\item le département informatique compte 40 à 50 personnes .
 			\item le terminal 1 est organiser en niveau du niveau 0 à 5 .
 		\end{itemize}
 		CDG2 et CDG3 est beaucoup plus simple car l'organisation de ces dernier sont fait en longueur alors que le CD1 est un Cylindre .
 		Les taches qu'a accomplit monsieur CAGNET lors de la gestion des tri bagages :
 		\begin{itemize}
 		 	\item Création d'une application qui permet de voit toutes le jetés en temps réels .
 		 	\item gestion des pannes possibles .
 		 	\item mise en place d'un automates pour les tapis 
 		 \end{itemize} 

 	\subsection{Technologie utilisée}
 	Le conférencier utilise principalement .NET et SQL lors de sa fonction .
 	\subsection{Avis personnel} 
 	Le parcours atypique ainsi les technologie que je connaissais peu ont rendue cette conférence très intéressante . De plus le ton moins formel à sue rendre la présentation plus détendue .